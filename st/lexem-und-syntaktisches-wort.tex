\definition{
Wörter sind die elementaren Einheiten, aus denen nach den Regeln der Syntax komplexere sprachliche Einheiten, nämlich Phrasen und schließlich Sätze, aufgebaut werden.
}

\noindent Die Wörter selbst können ebenfalls komplex sein. Die Regeln für den Aufbau komplexer Wörter beruhen aber auf anderen, eigenen Regeln, nämlich auf den Regeln der Morphologie.

Das Wort ist also ein grammatisches Konzept. Versuche, Wörter (auch) über die Bedeutung zu definieren, stoßen schnell an Grenzen. Man könnte etwa versucht sein, unter einem Wort eine elementare inhaltliche Einheit zu verstehen. Beim Abgleich mit der gramatischen Definition ergeben sich aber schnell Probleme:
\ex*{
	\item \hl{Tafel}: eine begriffliche Einheit. Grammtisch: ein Wort.
	\item \hl{Täfelchen}: eine komplexe begriffliche Einhei, deren Gesamtbedeutung von den Bestandteilen (Morphemen) \hl{Täfel-} und \hl{-chen} bestimmt wird. Grammtisch: ein Wort.
	\item \hl{Wandtafel}: ebenfalls eine komplexe begriffliche Einheit, die Bestandteile sind hier \hl{Wand-} und \hl{-tafel}. Grammtisch: ein Wort.
	\item \hl{Schwarzes} \hl{Brett}: eine oder zwei begriffliche Einheiten? Die häufige (aber nicht zwingende) Großschreibung des Adjektivs suggeriert Ersteres. Grammatisch: zwei Wörter.
}
Das Konzept der elementaren semantischen Einheit ist nicht sinnlos, es passt aber besser zum Konzept des Morphems \dudencite{grammatik10}{1018}.

Auch wenn geklärt ist, dass \enquote{Wort} ein grammatisches Konzept ist, sind noch Fragen offen. Siehe dazu die farbig hinterlegten \enquote{Wörter} in den folgenden beiden Sätzen:
\ex{
	\item Die Mauern des \hl{Turms} bestehen aus dicken Quadern.
	\item Die Touristen fotografieren die \hl{Türme.}
}
Wenn man die Unterschiede vor Augen hat, liegt als Konzept das syntaktische Wort zugrunde. Man kann dann beispielsweise feststellen: Das \enquote{Wort} \textit{Turms} im ersten Satz weist das Kasusmarkmal Genitiv auf und zeigt dies mit der Endung \textit{-s} auch an. Am \enquote{Wort} \textit{Türme} des zweiten Satzes kann man das Markmal Plural ablesen, die Verwendung im Satz lässt außerdem auf den Kasus Akkusativ schließen.
