\begin{syntax}{note}{}
\begin{itemize}
	\item auch \begin{itemize}
		\item Ich \verbhere \fillhere und \att{auch} (ein bisschen) \fillhere .
		\item \verbhere du/Sie \att{auch} \fillhere ?
		\item Ich präsentiere \att{auch} Projekte auf Englisch.
		\item Lernen Sie das Nomen \att{auch} im Plural.
	\end{itemize}
	\item jetzt \begin{itemize}
		\item \fillhere \verbhere \att{jetzt} \fillhere .
		\item Jetzt \fillhere \verbhere \fillhere .
		\item Ich brauche \att{jetzt} einen Kaffee.
	\end{itemize}
	\item und \begin{itemize}
		\item Ich \verbhere \fillhere gern und \verbhere \fillhere gern.
		\item Ich \verbhere \fillhere und \verbhere \fillhere gern.
	\end{itemize}
	\item in \begin{itemize}
		\item Ich wohne \att{in} \fillhere .
	\end{itemize}
	\item im \begin{itemize}
		\item Nomen im Plural/Singular.
	\end{itemize}
	\item viele \begin{itemize}
		\item Sie hat \att{viele} Besprechungen.
		\item Er liest \att{viele} Büchter.
		\item Sie schreibt \att{viele} E-Mails.
	\end{itemize}
	\item heute \begin{itemize}
		\item Frau Müller schreibt \att{heute} fünfzig E-Mails.
	\end{itemize}
	\item bei \begin{itemize}
		\item Peter arbeitet als Ingenieur bei Siemens.
	\end{itemize}
	\item am \begin{itemize}
		\item Felix spielt gern am Computer.
	\end{itemize}
	\item aber \begin{itemize}
			\item \att{Aber} ich lese auch viele Bücher auf Englisch.
			\item Du sprichst \att{aber} gut Deutsch.
	\end{itemize}
	\item für \begin{itemize}
			\item Sie bezahlen \att{für} Kaffee und Kuchen 8,60 Euro.
	\end{itemize}
	\item zusammen \begin{itemize}
			\item Sie gehen zusammen ins Museum.
	\end{itemize}
	\item dort \begin{itemize}
			\item \att{Dort} ist ein Tisch frei.
			\item Meine Frau arbeitet \att{dort} als Ärztin.
	\end{itemize}
	\item vielleicht \begin{itemize}
			\item Ich trinke \att{vielleicht} einen Tee oder einen Orangensaft oder \att{vielleicht} auch einen Kaffee \fillhere .
	\end{itemize}
	\item einmal \begin{itemize}
			\item \att{Einmal} Kaffee und \att{einmal} Mineralwasser.
	\end{itemize}
	\item das \begin{itemize}
			\item Ist \att{das} alles?
	\end{itemize}
	\item ach \begin{itemize}
			\item \att{Ach} nein, ich \fillhere .
	\end{itemize}
	\item also \begin{itemize}
			\item \att{Also} zwei Kaffee.
	\end{itemize}
	\item doch \begin{itemize}
			\item Ich nehme \att{doch} lieber einen Kaffee.
	\end{itemize}
\end{itemize}
\end{syntax}
