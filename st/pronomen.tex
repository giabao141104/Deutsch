%\begin{syntax}{Personelpronomen}{}
%\begin{figure}[H]
%\begin{tabular}{|l|l|l|l|l}
%	\hline
%	\multicolumn{2}{|l|}{} & Singular & Plural \\
%	\hline
%	\multicolumn{2}{|l|}{1. Person} & ich & wir \\
%	\hline
%	\multirow{2}{*}{2. Person} & informell & du & ihr \\
%	\cline{2-4}
%							   & formell & \multicolumn{2}{c|}{Sie} \\
%	\hline
%	\multicolumn{2}{|l|}{3. Person} & er/sie/es & sie \\
%	\hline
%\end{tabular}
%\end{figure}
%
%\begin{tabular}{llll}
%	der & \ssthere & $\rightarrow$ & er \\
%	die & \ssthere & $\rightarrow$ & sie \\
%	dar & \ssthere & $\rightarrow$ & es \\
%\end{tabular}
%\end{syntax}
%
%\begin{syntax}{Possessivpronomen}{}
%\begin{figure}[H]
%\begin{tabular}{|l|l|l|}
%	\hline
%	& Singular* & Plural \\
%	\hline
%	ich & mein(e) & meine \\
%	du & dein(e) & deine \\
%	\hline
%	er & sein(e) & seine \\
%	sie & ihr(e) & ihre \\
%	es & & \\
%	\hline
%	Sie & Ihr(e) & Ihre \\
%	\hline
%\end{tabular}
%\end{figure}
%
%*
%\begin{tabular}{lllll}
%	der & \ssthere & $\rightarrow$ & mein/dein/sein/ihr/Ihr & \ssthere \\
%	die & \ssthere & $\rightarrow$ & mein\att{e}/dein\att{e}/sein\att{e}/ihr\att{e}/Ihr\att{e} & \ssthere \\
%	dar & \ssthere & $\rightarrow$ & mein/dein/sein/ihr/Ihr & \ssthere
%\end{tabular}
%
%\begin{figure}[H]
%\begin{tabular}{|l|l|l|l|l|}
%	\hline
%	& \multicolumn{3}{c|}{Singular} & \multirow{2}{*}{Plural} \\
%	\cline{2-4}
%	& maskulin & feminin & neutral & \\
%	\hline
%	ich & mein & mein\att{e} & mein & mein\att{e} \\
%	\hline
%	du & dein & dein\att{e} & dein & dein\att{e} \\
%	\hline
%	er & sein & sein\att{e} & sein & sein\att{e} \\
%	sie & ihr & ihr\att{e} & ihr & ihr\att{e} \\
%	\hline
%	sie & ihr & ihr\att{e} & ihr & ihr\att{e} \\
%	\hline
%	Sie & Ihr & Ihr\att{e} & Ihr & Ihr\att{e} \\
%	\hline
%\end{tabular}
%\end{figure}
%\end{syntax}
%
%\begin{syntax}{Indefinitpronomen}{}
%\begin{itemize}
%	\item beide
%	\item bisschen \begin{itemize}
%			\item ein \att{bisschen} \begin{itemize}
%				\item Sprachen Sie ein \att{bisschen} Polnisch?
%			\end{itemize}
%		\end{itemize}
%	\item einige \begin{itemize}
%			\item Andreas liest \att{einige} Informationen über Frankfurt.
%		\end{itemize}
%	\item etwas \begin{itemize}
%			\item \att{Etwas} bestellen und bezahlen.
%			\item Möchest du \att{etwas} trinken?
%		\end{itemize}
%\end{itemize}
%\end{syntax}
