\begin{syntax}{Satzbau}{}
\begin{itemize}
	\item Aussagesatz:
		\begin{mdframed}
			Personalpronomen + \verbhere + \fillhere
		\end{mdframed}
	\item Ja-Nein-Fragen
		\begin{mdframed}
			\verbhere + Personalpronomen + \fillhere ?
		\end{mdframed}
	\item W-Fragen
		\begin{mdframed}
			\begin{tabular}{l}
				Woher \\
				Wo \\
				Wie \\
				Welche
			\end{tabular} + \verbhere + Personalpronomen ?
		\end{mdframed}
\end{itemize}

\begin{example}
\begin{tabular}{l|l|l|l}
	Position 1 & Position 2 & Mittelfeld & Satzende \\
	\bline
	Wir				& \att{möchten}	& gern								& \att{zahlen}. \\
	\att{Möchtest}	& du			& auch einen Kaffee					& \att{trinken}? \\
	\hline
	\att{Klaus}		& möchte		& \textbf{heute} einen Kräutertee	& trinken. \\
	\textbf{Heute}	& möchte		& \att{Klaus} einen Kräutertee		& trinken.
\end{tabular}
\end{example}
\begin{itemize}
	\item[Das Verb] Das konjugierte Verb steht auf Position 1 oder 2. \\
		Der Infinitiv steht am Satzende.
	\item[Das Subjekt] Das Subjekt steht oft auf Position 1, manchmal auf Position 3.
\end{itemize}
\end{syntax}
