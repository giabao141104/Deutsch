\begin{syntax}{Präpositionen}{}

\begin{itemize}
	\item auf \begin{itemize}
		\item Ich präsentiere auch Projekte \att{auf} Englisch.
		\item Ich schreibe viele E-Mails \att{auf} Englisch.
		\item Tereza liest Bücher \att{auf} Deutsch.
		\item \att{Auf} Platz 1 liegt Englisch.
		\item Das Subjekt steht oft \att{auf} Position 1, manchmal \att{auf} Position 3.
	\end{itemize}
	\item für \begin{itemize}
		\item Die Vorwahl \att{für} Deutschland ist 0049.
	\end{itemize}
	\item als \begin{itemize}
		\item Herr Faber arbeitet \att{als} Architekt.
		\item Peter arbeitet \att{als} Manager.
	\end{itemize}
	\item von \begin{itemize}
		\item Was sind Sie \att{von} Beruf?
		\item Das ist der Lippenstift \att{von} Marta?
		\item Was ist der Telefonnummber \att{von} Martina?
		\item Das ist die Tasche \att{von} Gabi.
		\item Der Lieblingsmaler \att{von} Andreas ist Claude Monet.
	\end{itemize}
	\item über \begin{itemize}
		\item Andreas und Petra reden \att{über} ihre Arbeit und ihre Familie.
		\item Andreas liest einige Informationen \att{über} Frankfurt.
	\end{itemize}
	\item mit \begin{itemize}
		\item \att{Mit} Milch und Zucker?
		\item \att{Mit} Milch und ohne Zucker.
	\end{itemize}
	\item ohne \begin{itemize}
		\item Mit Milch und \att{ohne} Zucker.
	\end{itemize}
\end{itemize}
\end{syntax}
