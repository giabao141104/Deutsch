\documentclass{report}
\usepackage{deutsch}

\begin{document}

\part{Phonetics}
\begin{phonetics}{Phonetik}{}
\begin{tabular}{ll}
	\ipa{S}					& sch		\\
	\ipa{Sp}				& sp		\\
	\ipa{a\textsubarch{I}}	& ei		\\
	\ipa{i:}				& ie
\end{tabular}
\end{phonetics}

\begin{phonetics}{Wortakzent}{}
\begin{itemize}
	\item \textbf{viele Verben:} Der Akzent ist auf der Stammsilbe. \\
		arbeiten, lesen, schreiben, hören, lernen, sprechen, malen, haben
	\item \textbf{Verben mit be-/ent-:} Der Akzent ist auf der Stammsilbe. \\
		bedienen, entwickeln
	\item \textbf{viele Verben mit unter-:} Der Akzent ist auf der Stammsilbe. \\
		unterrichten, untersuchen
	\item \textbf{Verben auf -ieren:} Der Akzent ist auf -ie-. \\
		konstruieren, studieren, formulieren
\end{itemize}
\begin{itemize}
	\item \textbf{viele Nomen:} Der Akzent ist auf der Stammsilbe. \\
		der Name, die Zeitung, die Flasche, die Seite, die Brille, der Schlüssel, die Lampe, der Kellner, die Lehrerin, der Künstler.
	\item \textbf{Komposita:} Der Akzent ist auf dem ersten Wort. \\
		der Fußball, das Lehrbuch, der Bildschirm, der Schreibtisch.
	\item \textbf{Fremdwörter:} Der Akzent ist oft auf der letzten Silbe. \\
		der Student, der Patient, das Medikament, der Assistent, der Ingenieur, die Präsentation, das Büro
\end{itemize}
\end{phonetics}


\part{Phonology}
\begin{phonology}{Das Alphabet}{}
\begin{tabular}{|c|c|c|c|c|c|c|c|c|c|}
	\hline
	A a & B b & C c & D d & E e & F f & G g & H h & I i & J j \\
	\ipa{a:} & \ipa{be:} & \ipa{tse:} & \ipa{de:} & \ipa{e:} & \ipa{Ef} & \ipa{ge:} & \ipa{ha:} & \ipa{i:} & \ipa{jOt} \\
	\hline
	K k & L l & M m & N n & O o & P p & Q q & R r & S s & T t \\
	\ipa{ka:} & \ipa{El} & \ipa{Em} & \ipa{En} & \ipa{o:} & \ipa{pe:} & \ipa{ku:} & \ipa{Er} & \ipa{Es} & \ipa{te:} \\
	\hline
	U u & V v & W w & X x & Y y & Z z & Ä ä & Ü ü & Ö ö & ß \\
	\ipa{u:} & \ipa{fa\textsubarch{O}} & \ipa{ve:} & \ipa{Iks} & \ipa{"YpsilOn} & \ipa{tsEt} & \ipa{E:} & \ipa{\o:} & \ipa{y:} & \ipa{Ests"Et} \\
	\hline
\end{tabular}
\end{phonology}

\begin{phonology}{Satzmelodie}{}
Melodie nach unten: \smd \\
Melodie nach oben: \smu

Fragen: \smu \\
Anworten: \smd
\end{phonology}


\part{Semantics}
%\begin{semantics}{Affix}{}
\begin{itemize}
	\item Präfix
		\begin{itemize}
			\item \href{https://de.wiktionary.org/wiki/ein-}{ein-}
		\end{itemize}
	\item Suffix
	\item Infix
	\item Interfix
	\item Zirkumfix
\end{itemize}
\end{semantics}

\begin{semantics}{Zahlen}{}
\href{./files/zahlen1.pdf}{zahlen1.pdf} \\
\href{./files/zahlen2.pdf}{zahlen2.pdf}
\begin{itemize}
	\item Komma
		\begin{mdframed}
			1,6 = eins Komma sechs
		\end{mdframed}
	\item Euro
		\begin{mdframed}
			8,60 = acht Euro sechzig
		\end{mdframed}
\end{itemize}
\end{semantics}

\begin{semantics}{Uhrzeit}{}
um \fillhere \\

\begin{mdframed}
	14.30 Uhr = 14 Uhr 30
\end{mdframed}

\begin{mdframed}
\begin{tabular}{lll}
	13.00 Uhr & = & eins \\
	16.10 Uhr & = & zehn nach vier \\
	15.15 Uhr & = & Viertel nach eins \\
	14.30 Uhr & = & halb drei \\
	18.45 Uhr & = & Viertel vor sieben \\
	20.55 Uhr & = & fünf vor neun
\end{tabular}
\end{mdframed}
\end{semantics}

\begin{semantics}{Tageszeiten}{}
um \fillhere \\

\begin{tabular}{lrlrl}
	der Morgen		& 6.00  & bis & 10.00 & Uhr \\
	der Vormittag	& 10.00 & bis & 12.00 & Uhr \\
	der Mittag		& 12.00 & bis & 14.00 & Uhr \\
	der Nachmittag	& 14.00 & bis & 18.00 & Uhr \\
	die Nacht		& 18.00 & bis & 22.00 & Uhr \\
	der Aabend		& 22.00 & bis & 6.00  & Uhr
\end{tabular}
\end{semantics}

\begin{semantics}{Dauer}{}
Er/sie/es dauert \fillhere

\begin{mdframed}
\begin{tabular}{lll}
	\textonehalf	& = & eine halbe Stunde \\
	1				& = & eine Stunde \\
	1\textonehalf	& = & anderthalb Stunden \\
	2				& = & zwei Stunden \\
	2\textonehalf	& = & zweieinhalb Stunden \\
	3				& = & drei Stunden \\
	3\textonehalf	& = & dreieinhalb Stunden
\end{tabular}
\end{mdframed}
\end{semantics}

%\begin{semantics}{Substantiv}{}
Singular und Plural
\begin{figure}[H]
\begin{tabular}{|l|l|l|}
	\hline
	Endung & Singular & Plural \\
	\hline
	-(e)n & die Lampe & die Lampen \\
		  & die Zeitung & die Zeitungen \\
	\hline
	-e (+Umlaut) & der Tisch & die Tische \\
				 & der Stuhl & die Stühle \\
	\hline
	-- (+Umlaut) & der Drucker & die Drucker \\
				 & der Apfel & die Äpfel \\
	\hline
	-s & das Handy & die Handys \\
	   & der Laptop & die Laptops \\
	\hline
	-er (+Umlaut) & das Bild & die Bilder \\
				  & das Buch & die Bücher \\
	\hline
\end{tabular}
\end{figure}

\begin{itemize}
	\item Stück + \begin{tabular}{l}
			Kuchen \\
			Schokoladenkuchen \\
			Brot
		\end{tabular}
\end{itemize}
\end{semantics}

%\begin{semantics}{Wünsche}{}
\begin{tabular}{|l|l|l|l|l}
	\hline
	\multicolumn{2}{|l|}{} & Singular & Plural \\
	\hline
	\multicolumn{2}{|l|}{1. Person} & möchte & möchten \\
	\hline
	\multirow{2}{*}{2. Person} & informell & möchtest & möchtet \\
	\cline{2-4}
							   & formell & \multicolumn{2}{c|}{möchten} \\
	\hline
	\multicolumn{2}{|l|}{3. Person} & möchte & möchten \\
	\hline
\end{tabular}

\begin{itemize}
	\item \att{möchte} mit Verb im Infinitiv \begin{itemize}
			\item \att{Möchten} Sie \att{zahlen}?
			\item \att{Möchtest} du einen Kaffee \att{trinken}?
		\end{itemize}
	\item \att{möchte} ohne Verb im Infinitiv \begin{itemize}
			\item Nein, ich \att{möchte} keinen Kaffee.
		\end{itemize}
\end{itemize}
\end{semantics}


\part{Pragmatics}
%\begin{pragmatics}{Wendungen im Alltag}{}
\begin{itemize}
	\item Guten Morgen!
	\item Guten Tag!
	\item Hallo!
	\item Guten Abend!
	\item Gute Nacht!
	\item Auf Wiedersehen!
	\item Tschüss!
	\item Bitte. -- Danke.
	\item Guten Appetit!
\end{itemize}
\end{pragmatics}


\part{Grammatik}

\chapter{Wörter und Wortbausteine}

	\section{Was ist ein Wort?}
		\begin{figure}[H]
\centering
%\begin{forest}
%for tree={grow'=0,folder,draw}
%[\hl{Wortart}
%	[flektierbar
%		[nach Tempus
%			[\hl{Verb}]
%		]
%		[nach Kasus
%			[fetes Genus
%				[\hl{Nomen}]
%			]
%			[veränderbares Genus
%				[nicht komparierbar
%					[\hl{Artikelwort} \hl{Pronomen}
%						[Personalpronomen]
%						[Reflexivpronomen]
%						[Possessivum]
%						[Demonstrativum]
%						[definiter Artikel]
%						[Relativum]
%						[Interrogativum]
%						[Indefinitum]
%						[indefiniter Artikel]
%					]
%				]
%				[komparierbar
%					[\hl{Adjektiv}]
%				]
%			]
%		]
%	]
%	[nicht flektierbar
%		[\hl{nicht flektierbare} \hl{Wortarten}
%			[Adverb]
%			[Präposition]
%			[Adjunktor]
%			[Subjunktion]
%			[Konjunktion]
%			[Partikel]
%		]
%	]
%]
%\end{forest}

% FOREST
\begin{forest}
[\hl{Wortart}
	[flektierbar
		[nach Tempus
					[\hl{Verb}, l*=4]
		]
		[nach Kasus
			[fetes \\ Genus, align=center, base=top
					[\hl{Nomen}, l*=3]
			]
			[veränderbares \\ Genus, align=center, base=top
				[nicht \\ komparierbar, align=center, base=top
					[\hl{Artikelwort} \\ \hl{Pronomen}, align=center, base=top
						[
							Personalpronomen \\
							Reflexivpronomen \\
							Possessivum \\
							Demonstrativum \\
							definiter Artikel \\
							Relativum \\
							Interrogativum \\
							Indefinitum \\
							indefiniter Artikel
						, align=center, base=top]
					]
				]
				[komparierbar
					[\hl{Adjektiv}, l*=1.5]
				]
			]
		]
	]
	[nicht \\ flektierbar, align=center, base=top
					[\hl{nicht flektierbare} \\ \hl{Wortarten}, align=center, base=top, l*=5
						[
							Adverb \\
							Präposition \\
							Adjunktor \\
							Subjunktion \\
							Konjunktion \\
							Partikel
						, align=center, base=top]
					]
	]
]
\end{forest}

% DUDEN STYLE
%\begin{tikzpicture}
%	\node {\key{Wortart}} [
%		level 1/.style={sibling distance=10cm},
%		level 2/.style={sibling distance=5cm},
%		level 3/.style={sibling distance=5cm},
%		level 4/.style={sibling distance=3cm},
%		level 5/.style={sibling distance=4cm},
%		level 6/.style={sibling distance=4cm,level distance = 4cm},
%	]
%    child {node {flektierbar}
%		child {node {nach Tempus}
%			child{
%				child{
%					child {node {\key{Verb}}}
%				}
%			}
%		}
%		child {node {nach Kasus}
%			child {node [align=center] {festes\\Genus}
%				child{
%					child {node {\key{Nomen}}}
%				}
%			}
%			child {node [align=center] {veränderbares\\Genus}
%				child {node [align=center] {nicht\\komparierbar}
%					child{node [align=center] {\key{Artikelwort}\\\key{Pronomen}}
%						child{node [align=center] {
%							Personalpronomen \\
%							Reflexivpronomen \\
%							Possessivum \\
%							Demonstrativum \\
%							definiter Artikel \\
%							Relativum \\
%							Interrogativum \\
%							Indefinitum \\
%							indefiniter Artikel
%							}
%						}
%					}
%				}
%				child {node {komparierbar}
%					child{node {\key{Adjektiv}}}
%				}
%			}
%		}
%	} 
%	child {node [align=center] {nicht\\flektierbar}
%		child{
%			child{
%				child{
%					child {node [align=center] {\key{nicht flektierbare}\\\key{Wortarten}}
%						child{node [align=center] {
%							Adverb \\
%							Präposition \\
%							Adjunktor \\
%							Subjunktion \\
%							Konjunktion \\
%							Partikel
%						}}
%					}
%				}
%			}
%		}
%	};
%\end{tikzpicture}
\end{figure}

		\newpage
		\subsection{Lexem und syntaktisches Wort}
		\definition{
Wörter sind die elementaren Einheiten, aus denen nach den Regeln der Syntax komplexere sprachliche Einheiten, nämlich Phrasen und schließlich Sätze, aufgebaut werden.
}

\noindent Die Wörter selbst können ebenfalls komplex sein. Die Regeln für den Aufbau komplexer Wörter beruhen aber auf anderen, eigenen Regeln, nämlich auf den Regeln der Morphologie.

Das Wort ist also ein grammatisches Konzept. Versuche, Wörter (auch) über die Bedeutung zu definieren, stoßen schnell an Grenzen. Man könnte etwa versucht sein, unter einem Wort eine elementare inhaltliche Einheit zu verstehen. Beim Abgleich mit der gramatischen Definition ergeben sich aber schnell Probleme:
\ex*{
	\item \hl{Tafel}: eine begriffliche Einheit. Grammtisch: ein Wort.
	\item \hl{Täfelchen}: eine komplexe begriffliche Einhei, deren Gesamtbedeutung von den Bestandteilen (Morphemen) \hl{Täfel-} und \hl{-chen} bestimmt wird. Grammtisch: ein Wort.
	\item \hl{Wandtafel}: ebenfalls eine komplexe begriffliche Einheit, die Bestandteile sind hier \hl{Wand-} und \hl{-tafel}. Grammtisch: ein Wort.
	\item \hl{Schwarzes} \hl{Brett}: eine oder zwei begriffliche Einheiten? Die häufige (aber nicht zwingende) Großschreibung des Adjektivs suggeriert Ersteres. Grammatisch: zwei Wörter.
}
Das Konzept der elementaren semantischen Einheit ist nicht sinnlos, es passt aber besser zum Konzept des Morphems \dudencite{grammatik10}{1018}.

Auch wenn geklärt ist, dass \enquote{Wort} ein grammatisches Konzept ist, sind noch Fragen offen. Siehe dazu die farbig hinterlegten \enquote{Wörter} in den folgenden beiden Sätzen:
\ex{
	\item Die Mauern des \hl{Turms} bestehen aus dicken Quadern.
	\item Die Touristen fotografieren die \hl{Türme.}
}
Wenn man die Unterschiede vor Augen hat, liegt als Konzept das syntaktische Wort zugrunde. Man kann dann beispielsweise feststellen: Das \enquote{Wort} \textit{Turms} im ersten Satz weist das Kasusmarkmal Genitiv auf und zeigt dies mit der Endung \textit{-s} auch an. Am \enquote{Wort} \textit{Türme} des zweiten Satzes kann man das Markmal Plural ablesen, die Verwendung im Satz lässt außerdem auf den Kasus Akkusativ schließen.

		\subsection{Wortarten}
		\subsection{Flexion}
		\subsection{Zum Begriff des Morphems}
	\section{Wortbildung}
	\section{Verb}
		\subsection{Übersicht}
		\subsection{Valenz und Bedeutung der Verben}
		\subsection{Flexion der Verben}
		\subsection{Wortbildung der Verben}
	\section{Nomen}
		\subsection{Übersicht}
		\subsection{Bedeutung der Nomen}
		\subsection{Valenz von Nomen}
		\subsection{Genus des Nomens}
		\subsection{Bildung der Pluralformen}
		\subsection{Kasusflexion des Nomen}
		\subsection{Wortbildung der Nomen}
	\section{Artikelwort und Pronomen}
		\subsection{Grundsätzliches zur Wortart}
		\subsection{Personalpronomen}
		\subsection{Reflexivpronomen}
		\subsection{Possessiva}
		\subsection{Demonstrativa}
		\subsection{Definiter Artikel}
		\subsection{Relativa}
		\subsection{Interrogativ}
		\subsection{Indefinita}
		\subsection{Indefiniter Artikel}
		\begin{syntax}{Demonstrativa}{}
\begin{itemize}
	\item \key{Das} als \sbj in Kopulasätzen. \\
%		\cite[377]{grammatik9}
%		\cite[1272]{grammatik10}
\end{itemize}
\end{syntax}

	\section{Adjektiv}
		\subsection{Inventar}
		\subsection{Bedeutung der Adjektive}
		\subsection{Gebrauch der Adjektive}
		\subsection{Valenz von Adjektiven}
		\subsection{Formenbildung}
		\subsection{Komparation von Adjektiven}
		\subsection{Wortbildung der Adjektive}
	\section{Adverb}
		\subsection{Syntaktische Eigenschaften}
		\subsection{Morphologische Eigenschaften}
		\subsection{Inventar}
		\subsection{Wortbildung der Adverbien}
	\section{Präposition}
		\subsection{Inventar}
		\subsection{Bedeutung und Funktion}
		\subsection{Kasusrektion}
		\subsection{Verschmelzung von Präposition und definitem Artikel}
	\section{Adjunktor, Subjunktion, Konjunktion}
	\section{Partikel}
	\section{Laut und Silbe}

\chapter{Sätze als Textbausteine}

	\section{Was ist ein Satz?}
		\subsection{Wortstellung: Abfolge von Satzgliedern und Prädikatsteilen}
		\begin{syntax}{Demonstrativa}{}
\begin{itemize}
	\item Das Feldermodell. \\
%		\cite[1339]{grammatik9}
%		\cite[20]{grammatik10}
	\item Die drei Satzformen. \\
%		\cite[1341]{grammatik9}
%		\cite[19]{grammatik10}
\end{itemize}
\end{syntax}

	\section{Was ist ein komplexer Satz?}
	\section{Parenthesen}
	\section{Auslassungen}
	\section{Zeit und Geltung}
		\subsection{Zeit und Tempus}
			\subsubsection{Grundlegendes: Aktionsart, Tempus und Zeitadverbiale}
			\subsubsection{Die Funktionen der indikativischen Tempora}
			\subsubsection{Tempussystem des Konjunktivs}
		\subsection{Geltung und Modus}
			\subsubsection{Aufforderungen und der Imperativ}
			\subsubsection{Modalverben}
			\subsubsection{Bedingungssatzgefüge}
			\subsubsection{Indirekte Rede- und Gedankenwiedergabe: Indikativ, Konjunktiv I, Konjunktiv II}
	\section{Syntaktische Negation}

%\begin{syntax}{Artikel}{}
	%\begin{tabular}{ll}
	%	bestimmter Artikel & der/die/das \\
	%	unbestimmter Artikel & ein \\
	%	negativer Artikel & kein
	%\end{tabular}

\begin{itemize}
	\item bestimmer Artikel
	\begin{mdframed}
		\begin{tabular}{ll}
			maskulin & \textbf{der} \\
			feminin & \textbf{die} \\
			neutral & \textbf{das}
		\end{tabular} + \ssthere
	\end{mdframed}

	\item unbestimmter Artikel
	\begin{mdframed}
		\begin{tabular}{lllll}
			der & \ssthere & $\rightarrow$ & ein & \ssthere \\
			die & \ssthere & $\rightarrow$ & ein\att{e} & \ssthere \\
			dar & \ssthere & $\rightarrow$ & ein & \ssthere
		\end{tabular}
	\end{mdframed}

	\item negativer Artikel
	\begin{mdframed}
		\begin{tabular}{lllll}
			der & \ssthere & $\rightarrow$ & kein & \ssthere \\
			die & \ssthere & $\rightarrow$ & kein\att{e} & \ssthere \\
			dar & \ssthere & $\rightarrow$ & kein & \ssthere
		\end{tabular}
	\end{mdframed}
\end{itemize}

Nomen und Artikel
\begin{figure}[H]
\begin{tabular}{l|ll|ll|ll|ll}
\hline
	& \multicolumn{6}{c|}{Singular} & \multirow{2}{*}{Plural} & \\
\cline{1-7}
	& \multicolumn{2}{l|}{maskulin} & \multicolumn{2}{l|}{feminin} & \multicolumn{2}{l|}{neutral} & & \\
\hline
	bestimmter Artikel & der & \ssthere & die & \ssthere & das & \ssthere & die & \ssthere \\
	unbestimmter Artikel & ein & \ssthere & eine & \ssthere & ein & \ssthere & -- & \ssthere \\
	negativer Artikel & kein & \ssthere & keine & \ssthere & kein & \ssthere & keine & \ssthere \\
\hline
\end{tabular}
\end{figure}
\end{syntax}

%\begin{syntax}{Präpositionen}{}

\begin{itemize}
	\item auf \begin{itemize}
		\item Ich präsentiere auch Projekte \att{auf} Englisch.
		\item Ich schreibe viele E-Mails \att{auf} Englisch.
		\item Tereza liest Bücher \att{auf} Deutsch.
		\item \att{Auf} Platz 1 liegt Englisch.
		\item Das Subjekt steht oft \att{auf} Position 1, manchmal \att{auf} Position 3.
	\end{itemize}
	\item für \begin{itemize}
		\item Die Vorwahl \att{für} Deutschland ist 0049.
	\end{itemize}
	\item als \begin{itemize}
		\item Herr Faber arbeitet \att{als} Architekt.
		\item Peter arbeitet \att{als} Manager.
	\end{itemize}
	\item von \begin{itemize}
		\item Was sind Sie \att{von} Beruf?
		\item Das ist der Lippenstift \att{von} Marta?
		\item Was ist der Telefonnummber \att{von} Martina?
		\item Das ist die Tasche \att{von} Gabi.
		\item Der Lieblingsmaler \att{von} Andreas ist Claude Monet.
	\end{itemize}
	\item über \begin{itemize}
		\item Andreas und Petra reden \att{über} ihre Arbeit und ihre Familie.
		\item Andreas liest einige Informationen \att{über} Frankfurt.
	\end{itemize}
	\item mit \begin{itemize}
		\item \att{Mit} Milch und Zucker?
		\item \att{Mit} Milch und ohne Zucker.
	\end{itemize}
	\item ohne \begin{itemize}
		\item Mit Milch und \att{ohne} Zucker.
	\end{itemize}
\end{itemize}
\end{syntax}

%\begin{syntax}{Temporaladverbien}{}
\begin{itemize}
	\item oft \begin{itemize}
		\item Er präsentiert \att{oft} Projekte.
		\item Ich bin beruflich \att{oft} in Polen.
	\end{itemize}
	\item noch \begin{itemize}
		\item Ich nehme \att{noch} ein Stück Schokoladenkuchen.
	\end{itemize}
	\item wirklich \begin{itemize}
		\item Möchtest du \att{wirklich} keinen Schokoladenkuchen?
	\end{itemize}
\end{itemize}
\end{syntax}

%\begin{syntax}{Adjektive}{}
\begin{mdframed}
	\adjhere + \ssthere
\end{mdframed}
\end{syntax}

%\begin{syntax}{Vergleichspartikel}{}
\begin{itemize}
	\item wie \begin{itemize}
		\item Sarah kocht auch gern, \att{wie} ihr Vater.
		\item so \adjhere \att{wie} \begin{itemize}
				\item Die Schweiz ist \att{so} groß \att{wie} die Niederlander.
			\end{itemize}
	\end{itemize}
\end{itemize}
\end{syntax}

%\begin{syntax}{Verben}{}
Konjugation die regelmäßige Verben
\begin{figure}[H]
\centering
\begin{tabular}{ll}
	ich & \ws{e} \\
	du & \ws{st} \\
	er/sie/es & \ws{t} \\
	wir & \ws{en} \\
	ihr & \ws{t} \\
	sie/Sie & \ws{en}
\end{tabular}
\end{figure}

Konjugation die unregelmäßigen Verben
\begin{itemize}
	\item Hilfsverb: sein
	\item Modalverb mit Infinitiv: mögen
\end{itemize}
\end{syntax}

%\begin{syntax}{Kasus}{}
Kasus
\begin{itemize}
	\item Nominativ
	\item Akkusativ
\end{itemize}

\begin{mdframed}
	\begin{tabular}{l|l|l}
		\textbf{Subjekt im Nominativ}	& \verbhere	& \textbf{Ergänzung im Akkusativ} \\
		\bline
		Ich							  	& brauche	& jetzt einen Kaffee. \\
		Mein Mann						& möchte	& einen Tee. \\
		Der Chef						& nimmt		& ein Wasser. \\
		Paul							& trinkt	& eine Limonade.
	\end{tabular}
\end{mdframed}

%\begin{figure}[H]
%\begin{tabular}{l|ll|ll|ll|ll}
%\hline
%	& \multicolumn{6}{c|}{Singular} & \multicolumn{2}{c}{\multirow{2}{*}{Plural}} \\
%\cline{2-7}
%	& \multicolumn{2}{c|}{maskulin} & \multicolumn{2}{c|}{feminin} & \multicolumn{2}{c|}{neutral} & & \\
%\hline
%	\multirow{3}{*}{Nominativ}	& der & \ssthere & die & \ssthere & das & \ssthere & die & \ssthere \\
%								& ein & \ssthere & ein\att{e} & \ssthere & ein & \ssthere & -- & \ssthere \\
%								& kein & \ssthere & kein\att{e} & \ssthere & kein & \ssthere & kein\att{e} & \ssthere \\
%\hline
%	\multirow{3}{*}{Akkusativ}	& \att{den} & \ssthere & die & \ssthere & das & \ssthere & die & \ssthere \\
%								& ein\att{en} & \ssthere & ein\att{e} & \ssthere & ein & \ssthere & -- & \ssthere \\
%								& kein\att{en} & \ssthere & kein\att{e} & \ssthere & kein & \ssthere & kein\att{e} & \ssthere \\
%\hline
%\end{tabular}
%\end{figure}

\begin{figure}[H]
\begin{tabular}{|l|l|l|l|l|}
\hline
	& \multicolumn{3}{c|}{Singular} & \multirow{2}{*}{Plural} \\
\cline{2-4}
	& maskulin & feminin & neutral & \\
\hline
	\multirow{3}{*}{Nominativ}	& der & die & das & die \\
								& ein & ein\att{e} & ein & -- \\
								& kein & kein\att{e} & kein & kein\att{e} \\
\hline
	\multirow{3}{*}{Akkusativ}	& \att{den} & die & das & die \\
								& ein\att{en} & ein\att{e} & ein & -- \\
								& kein\att{en} & kein\att{e} & kein & kein\att{e} \\
\hline
\end{tabular}
\end{figure}
\end{syntax}

%\begin{syntax}{Satzbau}{}
\begin{itemize}
	\item Aussagesatz:
		\begin{mdframed}
			Personalpronomen + \verbhere + \fillhere
		\end{mdframed}
	\item Ja-Nein-Fragen
		\begin{mdframed}
			\verbhere + Personalpronomen + \fillhere ?
		\end{mdframed}
	\item W-Fragen
		\begin{mdframed}
			\begin{tabular}{l}
				Woher \\
				Wo \\
				Wie \\
				Welche
			\end{tabular} + \verbhere + Personalpronomen ?
		\end{mdframed}
\end{itemize}

\begin{example}
\begin{tabular}{l|l|l|l}
	Position 1 & Position 2 & Mittelfeld & Satzende \\
	\bline
	Wir				& \att{möchten}	& gern								& \att{zahlen}. \\
	\att{Möchtest}	& du			& auch einen Kaffee					& \att{trinken}? \\
	\hline
	\att{Klaus}		& möchte		& \textbf{heute} einen Kräutertee	& trinken. \\
	\textbf{Heute}	& möchte		& \att{Klaus} einen Kräutertee		& trinken.
\end{tabular}
\end{example}
\begin{itemize}
	\item[Das Verb] Das konjugierte Verb steht auf Position 1 oder 2. \\
		Der Infinitiv steht am Satzende.
	\item[Das Subjekt] Das Subjekt steht oft auf Position 1, manchmal auf Position 3.
\end{itemize}
\end{syntax}

%\begin{syntax}{Was passt?}{}
\begin{itemize}
	\item \begin{tabular}{l}
			Gitarre \\
			Klavier \\
			Tennis \\
			Fußball \\
			Volleyball
		\end{tabular} + \textbf{spielen}
	\item \textit{Sprache} + \textbf{sprechen}
	\item \begin{tabular}{l}
			\textit{Sprache} \\
			schwimmen \\
			sprechen \\
			kochen
		\end{tabular} + \textbf{lernen}
	\item \begin{tabular}{l}
			Gymnastik \\
			Musik
		\end{tabular} + \textbf{machen}
	\item \begin{tabular}{l}
			Gymnastik \\
			Musik \\
			Design \\
			Jura
		\end{tabular} + \textbf{studieren}
	\item Gäste + \textbf{bedien}
	\item Patienten + \textbf{untersuchen}
	\item \begin{tabular}{l}
			E-Mails \\
			Bücher
		\end{tabular} + \textbf{schreiben}
	\item \begin{tabular}{l}
			E-Mails \\
			Besprechungen
		\end{tabular} + \textbf{haben}
	\item Computerspiele + \textbf{entwickeln}
	\item Kinder + \textbf{unterrichten}
	\item Bücher + \textbf{lesen}
	\item Projekte + \textbf{präsentieren}
	\item Solarautos + \textbf{konstruieren}
	\item Kaffee + \textbf{trinken} 
	\item Schokoladenkuchen + \textbf{essen}
	\item \fillhere Euro + \textbf{bezahlen}
	\item eine Konferenze + \textbf{besuchen}
	\item über der/die/das \ssthere + \textbf{reden}
	\item ins Museum + \textbf{gehen}
	\item ein Hotelzimmer + \textbf{suchen}
	\item im Hotel + \textbf{übernachten}
	\item Informationen über \fillhere + \textbf{lesen}
\end{itemize}
\end{syntax}

%\begin{syntax}{note}{}
\begin{itemize}
	\item auch \begin{itemize}
		\item Ich \verbhere \fillhere und \att{auch} (ein bisschen) \fillhere .
		\item \verbhere du/Sie \att{auch} \fillhere ?
		\item Ich präsentiere \att{auch} Projekte auf Englisch.
		\item Lernen Sie das Nomen \att{auch} im Plural.
	\end{itemize}
	\item jetzt \begin{itemize}
		\item \fillhere \verbhere \att{jetzt} \fillhere .
		\item Jetzt \fillhere \verbhere \fillhere .
		\item Ich brauche \att{jetzt} einen Kaffee.
	\end{itemize}
	\item und \begin{itemize}
		\item Ich \verbhere \fillhere gern und \verbhere \fillhere gern.
		\item Ich \verbhere \fillhere und \verbhere \fillhere gern.
	\end{itemize}
	\item in \begin{itemize}
		\item Ich wohne \att{in} \fillhere .
	\end{itemize}
	\item im \begin{itemize}
		\item Nomen im Plural/Singular.
	\end{itemize}
	\item viele \begin{itemize}
		\item Sie hat \att{viele} Besprechungen.
		\item Er liest \att{viele} Büchter.
		\item Sie schreibt \att{viele} E-Mails.
	\end{itemize}
	\item heute \begin{itemize}
		\item Frau Müller schreibt \att{heute} fünfzig E-Mails.
	\end{itemize}
	\item bei \begin{itemize}
		\item Peter arbeitet als Ingenieur bei Siemens.
	\end{itemize}
	\item am \begin{itemize}
		\item Felix spielt gern am Computer.
	\end{itemize}
	\item aber \begin{itemize}
			\item \att{Aber} ich lese auch viele Bücher auf Englisch.
			\item Du sprichst \att{aber} gut Deutsch.
	\end{itemize}
	\item für \begin{itemize}
			\item Sie bezahlen \att{für} Kaffee und Kuchen 8,60 Euro.
	\end{itemize}
	\item zusammen \begin{itemize}
			\item Sie gehen zusammen ins Museum.
	\end{itemize}
	\item dort \begin{itemize}
			\item \att{Dort} ist ein Tisch frei.
			\item Meine Frau arbeitet \att{dort} als Ärztin.
	\end{itemize}
	\item vielleicht \begin{itemize}
			\item Ich trinke \att{vielleicht} einen Tee oder einen Orangensaft oder \att{vielleicht} auch einen Kaffee \fillhere .
	\end{itemize}
	\item einmal \begin{itemize}
			\item \att{Einmal} Kaffee und \att{einmal} Mineralwasser.
	\end{itemize}
	\item das \begin{itemize}
			\item Ist \att{das} alles?
	\end{itemize}
	\item ach \begin{itemize}
			\item \att{Ach} nein, ich \fillhere .
	\end{itemize}
	\item also \begin{itemize}
			\item \att{Also} zwei Kaffee.
	\end{itemize}
	\item doch \begin{itemize}
			\item Ich nehme \att{doch} lieber einen Kaffee.
	\end{itemize}
\end{itemize}
\end{syntax}

%\begin{syntax}{Konjunktion}{}
\begin{itemize}
	\item \fillhere so \fillhere wie \fillhere .
\end{itemize}
\end{syntax}


\part{Discourse}
% UNDOCUMENTED
% auch, noch, ein bisschen, auf + SPRACHE, sehr
% ..., wie ...
% wie lange ... schon ...
\begin{discourse}{Some expression}{}
\begin{tabularx}{\linewidth}{XV{3}X}
	\multicolumn{2}{l}{\textbf{To expression uncertainty}} \\
	\bline
	Ich glaube \fillhere . & \\
	\multicolumn{2}{l}{}
\end{tabularx}

\begin{tabularx}{\linewidth}{XV{3}X}
	\multicolumn{2}{l}{\textbf{Positive Reaktionen}} \\
	\bline
	Ich auch. & \\
	Interessant! & \\
	Wirklich? & \\
	Toll! & \\
	Super! & \\
	Ja, das stimmt! & \\
	Prima Idee! & \\
	Das machen wir! & \\
	\multicolumn{2}{l}{}
\end{tabularx}

\begin{tabularx}{\linewidth}{XV{3}X}
	\multicolumn{2}{l}{\textbf{}} \\
	\bline
	Und Sie? & \\
	\multicolumn{2}{l}{}
\end{tabularx}

\begin{tabularx}{\linewidth}{XV{3}X}
	\multicolumn{2}{l}{\textbf{Zeitpunkt}} \\
	\bline
	Wie spät ist es? & Es ist \kontext{Zeitpunkt}. \\
\ro	Wann \kjg{beginnen} \fillhere & \fillhere \kjg{beginnen} um \kontext{Zeitpunkt}. \\
\ro	Wan ist \fillhere zu Ende? & \fillhere \kjg{sein} um \kontext{Zeitpunkt} zu Ende. \\
	Wie viel Uhr \fillhere ? & \\
	\multicolumn{2}{l}{}
\end{tabularx}

\begin{tabularx}{\linewidth}{XV{3}X}
	\multicolumn{2}{l}{\textbf{Dauer}} \\
	\bline
	& \fillhere \kjg{dauern} \kontext{Dauer} \\
	\multirow{-2}{\linewidth}{Wie lange \kjg{dauern} \fillhere ?} & \fillhere \kjg{gehen} von \kontext{Zeitpunkt} bis \kontext{Zeitpunkt}. \\
\ro	Von wann bis wan \kjg{haben} \sbj Zeit? & Von \kontext{Zeitpunkt} bis \kontext{Zeitpunkt}. \\
\ro	Bis wann \kjg{haben} \sbj Zeit? & Bis \kontext{Zeitpunkt}. \\
	Bis wann \verbhere \sbj? & \sbj \verbhere bis \kontext{Zeitpunkt}. \\
	\multicolumn{2}{l}{}
\end{tabularx}
\end{discourse}

\begin{discourse}{Jemanden begrüßen und verabschieden}{}
\begin{tabularx}{\linewidth}{XV{3}X}
	\multicolumn{2}{l}{\textbf{Wichtige Wendungen im Alltag}} \\
	\bline
	Guten Morgen! & \\
\ro	Guten Tag! & \\
	Hallo! & \\
\ro	Guten Abend! & \\
	Gute Nacht! & \\
\ro	Auf Wiedersehen! & \\
	Tschüss! & \\
\ro	Bitte. & Danke. \\
	Guten Appetit & \\
\ro												& Dank, gut. \\
\ro \multirow{-2}{*}{Wie \kjg{gehen} es \sbj ?} & Es gehts. \\
	Hallo \fillhere , so eine Überraschung! & \\
	\multicolumn{2}{l}{}
\end{tabularx}
\end{discourse}

\begin{discourse}{Person}{person}
\begin{tabularx}{\linewidth}{XV{3}X}
	\multicolumn{2}{l}{\textbf{Name}} \\
	\bline
	Hallo! & Guten Tag! \\
\ro	Wie heißen Sie? & Ich heiße \fillhere . \\
\ro	Wer sind Sie? & Ich bin \fillhere . \\
\ro	Wie ist Ihr Name? & Mein Name ist \fillhere . \\
	\multicolumn{2}{l}{}
\end{tabularx}

\begin{tabularx}{\linewidth}{XV{3}X}
	\multicolumn{2}{l}{\textbf{Adresse}} \\
	\bline
	Wie ist Ihre Adresse? & Meine Adresse ist \fillhere . \\
	\multicolumn{2}{l}{}
\end{tabularx}

\begin{tabularx}{\linewidth}{XV{3}X}
	\multicolumn{2}{l}{\textbf{Telefon}} \\
	\bline
	Was ist Ihr Telefonnummer? & Meine Telefonnummer ist \fillhere . \\
\ro	Wie ist die Telefonnummber von \fillhere? & \\
\ro	Welche Telefonnummer \kjg{haben} \sbj ? & \multirow{-2}{\linewidth}{Die Telefonnummber von \fillhere ist \fillhere .} \\
	\multicolumn{2}{l}{}
\end{tabularx}

\begin{tabularx}{\linewidth}{XV{3}X}
	\multicolumn{2}{l}{\textbf{Geburtsort}} \\
	\bline
	\multicolumn{2}{l}{}
\end{tabularx}

\begin{tabularx}{\linewidth}{XV{3}X}
	\multicolumn{2}{l}{\textbf{Alter}} \\
	\bline
	\sbj \kjg{sein} \kontext{Anzalh} Jahr alt. & \\
	\multicolumn{2}{l}{}
\end{tabularx}

\begin{tabularx}{\linewidth}{XV{3}X}
	\multicolumn{2}{l}{\textbf{Geschlecht}} \\
	\bline
	\multicolumn{2}{l}{}
\end{tabularx}

\begin{tabularx}{\linewidth}{XV{3}X}
	\multicolumn{2}{l}{\textbf{Familienstand/Familie}} \\
	\bline
	\kjg{sein} \sbj verheiratet/geschieden/ledig/Single? & \sbj \kjg{sein} verheiratet/geschieden/ledig/Single. \\
\ro	\sbj \kjg{wohnen} mit \kontext{Person} zusammen. & \\
\ro	\sbj \kjg{wohnen}/\kjg{leben} allein. & \\
	\kjg{Haben} \sbj Kinder? & \sbj \kjg{haben} \kontext{Anzahl} \kjg{Kind}. \\
	\multicolumn{2}{l}{}
\end{tabularx}

\begin{tabularx}{\linewidth}{XV{3}X}
	\multicolumn{2}{l}{\textbf{Persönliche Beziehungen}} \\
	\bline
	\multicolumn{2}{l}{}
\end{tabularx}

\begin{tabularx}{\linewidth}{XV{3}X}
	\multicolumn{2}{l}{\textbf{Statsangehörigkeit/Nationalität/Herkunft}} \\
	\bline
	Woher kommen Sie? & Ich komme aus \fillhere . \\
\ro	Wo wohnen Sie? & Ich wohne in \fillhere . \\
															 & Ja, \sbj \kjg{wohnen} in \kontext{Stadt}. \\
	\multirow{-2}{*}{\kjg{Wohnen} \sbj in \kontext{Stadt} ?} & Nein, \sbj \kjg{wohnen} in \kontext{Stadt}. \\
\ro	Wo ist \kontext{Stadt}? & \kontext{Stadt} ist in \kontext{Land}. \\
	\sbj \kjg{wohnen} (doch) jetzt in \kontext{Land}, oder? & \sbj \kjg{arbeiten} dort als \kontext{Beruf} . \\
	\multicolumn{2}{l}{}
\end{tabularx}

\begin{tabularx}{\linewidth}{XV{3}X}
	\multicolumn{2}{l}{\textbf{Aussehen}} \\
	\bline
	\multicolumn{2}{l}{}
\end{tabularx}

\begin{tabularx}{\linewidth}{XV{3}X}
	\multicolumn{2}{l}{\textbf{Gewohnheiten/Tagesablauf}} \\
	\bline
	\multicolumn{2}{l}{}
\end{tabularx}

\begin{tabularx}{\linewidth}{XV{3}X}
	\multicolumn{2}{l}{\textbf{Kennzeichen}} \\
	\bline
	Welches Kennzeichen hat Ihre Auto? & Mein Auto hat das Kennzeichen \fillhere . \\
	Welches Kennzeichen hat das Audo von \fillhere ? & Das Auto von \fillhere hat das Kennzeichen \fillhere . \\
	\multicolumn{2}{l}{}
\end{tabularx}
\end{discourse}

\begin{discourse}{Reisen/Verkehr}{einkaufen}
\begin{tabularx}{\linewidth}{XV{3}X}
	\multicolumn{2}{l}{\textbf{privater und öffentlicher Verkehr}} \\
	\bline
	\multicolumn{2}{l}{}
\end{tabularx}

\begin{tabularx}{\linewidth}{XV{3}X}
	\multicolumn{2}{l}{\textbf{Reisen}} \\
	\bline
	& Ja, da war ich schon. \\
	& Nein, da war ich noch nicht. \\
	\multirow{-3}{\linewidth}{Waren Sie schon mal in \kontext{Ort}?} & Nein. Ich war schon in \kontext{Ort}, aber ich war noch nicht in \kontext{Ort}. \\
\ro	Wie war es? & Es war herrlich/schön/kalt,warm. \\
	& Ja, es ist \adjhere \\
	\multirow{-2}{\linewidth}{Ist es \adjhere in \kontext{Ort}?} & Keine Ahnung. Ich war auch noch nicht in \kontext{Ort}. \\
	\multicolumn{2}{l}{}
\end{tabularx}

\begin{tabularx}{\linewidth}{XV{3}X}
	\multicolumn{2}{l}{\textbf{Unterkunft}} \\
	\bline
	\multicolumn{2}{l}{}
\end{tabularx}

\begin{tabularx}{\linewidth}{XV{3}X}
	\multicolumn{2}{l}{\textbf{Gepäck}} \\
	\bline
	\multicolumn{2}{l}{}
\end{tabularx}

\begin{tabularx}{\linewidth}{XV{3}X}
	\multicolumn{2}{l}{\textbf{Weg fragen}} \\
	\bline
	Ich suche \kjg{ein} \fillhere . Ich möchte \fillhere . & \\
	\multicolumn{2}{l}{}
\end{tabularx}

\begin{tabularx}{\linewidth}{XV{3}X}
	\multicolumn{2}{l}{\textbf{Hotel}} \\
	\bline
	Herzlich willkommen im Hotel \fillhere & Ich möchte bitte ein Einzelzimmer/Doppelzimmer. \\
	\hline
	Haben Sie noch ein Zimmer frei? & \\
\ro	& Ja, gerne. \\
\ro	\multirow{-2}{\linewidth}{Ja, möchten Sie ein Einzelzimmer?} & Nein, ich möchte ein Doppelzimmer. \\
	Haben Sie eine Reservierung? & Nein, ich habe heine Reservierung. \\
	\hline
	Wie lange möchten Sie bleiben? & Eine Nacht./\fillhere Nächte. \\
	\hline
	Ist das Zimmer für eine Nacht? & Ja, für eine Nacht.
									Moment bitte. Ja. Wir haben noch ein Einzelzimmer für Sie. \\
\ro	Was kostet das Zimmer? &  \\
\ro	Wie viel kostet das Zimmer & \multirow{-2}{\linewidth}{Das Zimmer kostet \kontext{Preis} pro Nacht.} \\
	Mit Frühstück? & Nein, der Preis ist ohne Frühstück.
					Das Frühstück kostet \kontext{Preis} extra. \\
\ro	Gibt es WLAN? & Natürlich. Der Code steht hier auf der Zimmerkarte. \\
	Gut. Ich nehme das Zimmer. & Ich brauche noch Ihre persönlichen Angaben hier auf dem Formular. \\
\ro	Hier ist Ihre Zimmerkarte, das ist der Code für das WLAN. & \\
\ro	Ihre Zimmernummer ist die \kontext{Zimmernummer}. & \\
	Vielen Dank. & Schönen Aufenthalt! \\
	\multicolumn{2}{l}{}
\end{tabularx}
\end{discourse}

\begin{discourse}{Essen und Drinken}{essen-trinken}
\begin{tabularx}{\linewidth}{XV{3}X}
	\multicolumn{2}{l}{\textbf{Nahrungsmittel}} \\
	\bline
	\multicolumn{2}{l}{}
\end{tabularx}

\begin{tabularx}{\linewidth}{XV{3}X}
	\multicolumn{2}{l}{\textbf{Mahlzeiten}} \\
	\bline
	\multicolumn{2}{l}{}
\end{tabularx}

\begin{tabularx}{\linewidth}{XV{3}X}
	\multicolumn{2}{l}{\textbf{Speisen und Getränke}} \\
	\bline
	Am liebsten \kjg{essen}/\kjg{trinken} \sbj \fillhere . & \\
\ro	Was trinken/essen die Menschen in Ihrem Heimatland zum \kontext{Zeit}? & Ich komme aus \kontext{Land}. In \kontext{Land} trinkt/isst man zum \kontext{Zeit} \fillhere . \\
\ro	Welche Lebensmittel isst man oft in Ihrem Heimatland? & \\
\ro Welche Produkte isst man selten/nie? & \\
	Was essen/trinken Sie gern? & \\
	Was essen/trinken Sie am liebsten? & \\
	Was essen/trinken Sie oft/viel? & \\
	Was essen/trinken Sie selten/wenig? & \multirow{-4}{\linewidth}{Ich esse/trinke gern \fillhere .} \\
\ro	Was mögen Sie nicht? & \\
	Mögen Sie \fillhere ? & \\
	Essen/Trinken Sie gern/viel \fillhere ? & \makecell[l]{Ja, ich esse/trinke gern/vie \fillhere . \\ Nein, ich esse/trinke nicht gern/viel \fillhere .} \\
\ro Wann essenn/trinken Sie \fillhere ? & Ich esse/trinke \kontext{Zeit} \fillhere . \\
	Wo essen/trinken Sie \fillhere ? & \fillhere essen/trinken ich \fillhere . \\
\ro Wie schmeckt \fillhere ? & Es schmekt \fillhere . Es ist \fillhere \\
	\multicolumn{2}{l}{}
\end{tabularx}

\begin{tabularx}{\linewidth}{XV{3}X}
	\multicolumn{2}{l}{\textbf{Lokale (Restaurant, Café)}} \\
	\bline
	Möchten Sie etwas essen/trinken? & \\
	Dort ist ein Tisch frei. & \\
	\hline
	Was nehmen Sie?					& Ich nehme (auch) \fillhere . \\
	Was möchten Sie trinken/essen?	& Ich möchte (gern) \fillhere . \\
	Was trinken Sie?				& Ich trinke \fillhere . \\
	Was essen Sie?					& Ich esse \fillhere . \\
									& Ich brauche (jetzt) \fillhere . \\
									& Ich hätte gern \fillhere . \\
									& Ich möchte bitte \fillhere . \\
	\hline
	Möchten Sie (auch) \fillhere?					& Nein, danke. \\
	\ro Möchten Sie wirklich \kjg{kein} \fillhere?	& Ach doch, ich nehme auch \kjg{ein} \fillhere . \\
	\hline
																			  & Nein, ich möchte \kjg{kein} \fillhere . \\
	\multirow{-2}{\linewidth}{Möchten Sie \kjg{ein} \fillhere essen/trinken?} & Ich esse/trinke lieber \kjg{ein} \fillhere . \\
\ro	Ist das alles?								& Nein, ich nehme noch \fillhere . \\
\ro	Also einen \fillhere und \fillhere . & Ja, bitte. \\
	\hline
	Wie viel kostet \fillhere ?	& \fillhere kostet \kontext{Preis}. \\
	Was kostet \fillhere ?		& \fillhere kostet \kontext{Preis}. \\
\ro Dann nehme ich auch noch \fillhere . & Gerne. Vielen Dank. \\
	\hline
	Ich brauche \fillhere . & Tut mir leid, ich habe \kjg{kein} \fillhere . \\
	Ohne \fillhere kann ich nicht essen/trinken. & \\
\ro	Kann ich bitte \fillhere haben? & Tut mir leid, ich habe \kjg{kein} \fillhere . \\
\ro	Ich hätte gern \fillhere . & Einen Moment bitte, ich hole \fillhere . \\
\ro	Ich möchte bitte \fillhere . & Ja, gerne. \\
	\hline
	Wie heißt das Restaurant? & Das Restaurant heßt \fillhere . \\
\ro	Wann ist das Restaurant geöffnet? & \makecell[l]{Es ist von \kontext{Wochentag} bis \kontext{Wochentag} geöffnet. \\
								Es ist von \kontext{Zeit} bis \kontext{Zeit} geöffnet.} \\
	Was kann man dort essen? & \makecell[l]{Man kann dort \fillhere essen . \\
									Auf der Speisekarte stehen: \fillhere . \\
									Das Angebot ist groß/klein.} \\
\ro	Was kostet ein Abendessen? & \fillhere kostet \fillhere . \\
\ro	Was kostet der Wein? & Ein Glas/Flasche Wein kostet \fillhere . \\
	\hline
	Restaurant \fillhere, guten Tag. & Ja, guten Tag, ich möchte gern einen Tisch für \fillhere Personen reservieren. \\
\ro	Wann möchten Sie kommen? & Am \kontext{Wochentag} um \kontext{Zeit}. \\
	\makecell[l]{Am \kontext{Wochentag} um \kontext{Zeit}, für \fillhere Personen. \\
		Ja, das geht, wir haben noch einen Tisch frei. \\
		Wie ist Ihr Name?} & Mein Name ist \fillhere . \\
\ro Können Sie Ihren Name bitte buchstabieren? & Gern, \fillhere . \\
	Vielen Dank. Wir sehen Sie am \kontext{Wochentag} . & Danke. Bis \kontext{Wochentag}. Auf Wiederhören. \\
	\multicolumn{2}{l}{}
\end{tabularx}

\begin{example}
\begin{tabularx}{\linewidth}{XV{3}X}
												& Ich brauche jetzt einen Kaffee. \\
	\multirow{-2}{\linewidth}{Was möchten Sie trinken?}	& Ich möchte bittet einen Kaffee. \\
\ro	Mit Milch und Zucker?						& Mit viel Milch und ohne Zucker. \\
\end{tabularx}
\end{example}
\end{discourse}

\begin{discourse}{Einkaufen/Gebrauchsartikel}{einkaufen}
\begin{tabularx}{\linewidth}{XV{3}X}
	\multicolumn{2}{l}{\textbf{Geschäfte}} \\
	\bline
	\multicolumn{2}{l}{}
\end{tabularx}

\begin{tabularx}{\linewidth}{XV{3}X}
	\multicolumn{2}{l}{\textbf{Preis/Bezahlen}} \\
	\bline
	\sbj bezahlen/zahlen für \fillhere \kontext{Preis}. & \\
	\hline
	\sbj \kontext{möchten} (gern) zahlen. & Das macht zusammen \kontext{Preis}. \\
	\hline
	Zusammen oder getrennt?		& Zusammen. \\
	\ro							& \fillhere und \fillhere, das macht \kontext{Preis}. \\
	\ro							& Das macht zusammaen \kontext{Preis}. \\
	Bitte. & Vielen Dank. Auf Wiedersehen. \\
	\hline
	& Bar. \\
	\multirow{-2}{\linewidth}{Wie zahlen Sie? Bar oder mit Kreditkarte?} & Mit Kreditkarte. \\
\ro	Was ist besonders teuer? & \\
\ro	Was kostet wenig Geld? & \\
	\multicolumn{2}{l}{}
\end{tabularx}

\begin{tabularx}{\linewidth}{XV{3}X}
	\multicolumn{2}{l}{\textbf{Lebensmittel}} \\
	\bline
	\multicolumn{2}{l}{}
\end{tabularx}

\begin{tabularx}{\linewidth}{XV{3}X}
	\multicolumn{2}{l}{\textbf{Kleidung}} \\
	\bline
	\multicolumn{2}{l}{}
\end{tabularx}
\end{discourse}

\begin{discourse}{Erziehung/Ausbildung/Lernen}{person}
\begin{tabularx}{\linewidth}{XV{3}X}
	\multicolumn{2}{l}{\textbf{Kinderbetreuung}} \\
	\bline
	\multicolumn{2}{l}{}
\end{tabularx}

\begin{tabularx}{\linewidth}{XV{3}X}
	\multicolumn{2}{l}{\textbf{Schule}} \\
	\bline
	\multicolumn{2}{l}{}
\end{tabularx}

\begin{tabularx}{\linewidth}{XV{3}X}
	\multicolumn{2}{l}{\textbf{Sprachen lernen}} \\
	\bline
														   & \sbj \kjg{sprechen} \kontext{Sprache(n)} . \\
	\multirow{-2}{\linewidth}{Welche Sprachen \kjg{sprechen} \sbj?} & \sbj \kjg{lernen} jetzt \kontext{Sprache(n)} . \\
\ro												  & Meine Muttersprache ist \kontext{Sprache} . \\
\ro	\multirow{-2}{\linewidth}{Was ist Ihre Muttersprache?} & \sbj \kjg{sprechen} \kontext{Sprache} als Muttersprache. \\
	\multicolumn{2}{l}{}
\end{tabularx}
\end{discourse}

\begin{discourse}{Arbeit/Beruf}{person}
\begin{tabularx}{\linewidth}{XV{3}X}
	\multicolumn{2}{l}{\textbf{Arbeit}} \\
	\bline
	\multicolumn{2}{l}{}
\end{tabularx}

\begin{tabularx}{\linewidth}{XV{3}X}
	\multicolumn{2}{l}{\textbf{Beruf}} \\
	\bline
	\sbj \kjg{studieren} \fillhere & Später \kjg{arbeiten} \sbj als \kontext{Beruf}. \\
\ro	Was \kjg{sein} \sbj von Beruf? & \sbj \kjg{sein} \fillhere . \\
\ro	Was \kjg{machen} \sbj beruflich? & \sbj \kjg{arbeiten} als \fillhere . \\
\ro	& \sbj \kjg{sein} beruflich in \kontext{Ort}. \\
	\kjg{Sein} \sbj \kontext{Beruf}? & Ja/Nein, \sbj \kjg{sein} \fillhere . \\
\ro	Was \kjg{machen} \sbj als \kontext{Beruf}? & \sbj \fillhere . \\
																			& \sbj \kjg{sein} beruflich hier. \\
	\multirow{-2}{\linewidth}{Was \kjg{machen} \sbj hier in \kontext{Ort}?} & \sbj \kjg{sein} beruflich in \kontext{Ort}. \\
\ro	Wo \kjg{arbeiten} \sbj? & \sbj \kjg{arbeiten} bei \fillhere als \kontext{Beruf} . \\
	\multicolumn{2}{l}{}
\end{tabularx}
\end{discourse}

\begin{discourse}{Freizeit/Unterhaltung}{person}

\begin{tabularx}{\linewidth}{XV{3}X}
	\multicolumn{2}{l}{\textbf{Interessen}} \\
	\bline
	Was \kjg{machen} \sbj gern? & \sbj \verbhere gern \fillhere . \\
\ro													  & Ja, \sbj \verbhere gern \fillhere . \\
\ro													  & Ich \verbhere nicht so gern \fillhere . \\
\ro													  & Nein, \sbj \verbhere nicht gern \fillhere . \\
\ro	\multirow{-4}{*}{\verbhere \sbj gern \fillhere ?} & \sbj \verbhere lieber \fillhere . \\
	\hline
	& \kjg{Mein} Lieblings-\fillhere ist \fillhere . \\
	\multirow{-2}{\linewidth}{Wer ist \kjg{dein/Ihr} Lieblings-\fillhere ?} & Ich \kjg{haben} \kjg{kein} Lieblings-\fillhere . \\
	\multicolumn{2}{l}{}
\end{tabularx}

\begin{tabularx}{\linewidth}{XV{3}X}
	\multicolumn{2}{l}{\textbf{Sport treiben}} \\
	\bline
	\multicolumn{2}{l}{}
\end{tabularx}

\begin{tabularx}{\linewidth}{XV{3}X}
	\multicolumn{2}{l}{\textbf{Radio/Fernsehen}} \\
	\bline
	\multicolumn{2}{l}{}
\end{tabularx}

\begin{tabularx}{\linewidth}{XV{3}X}
	\multicolumn{2}{l}{\textbf{Internet}} \\
	\bline
	\multicolumn{2}{l}{}
\end{tabularx}

\begin{tabularx}{\linewidth}{XV{3}X}
	\multicolumn{2}{l}{\textbf{Lektüre/Presse}} \\
	\bline
	\multicolumn{2}{l}{}
\end{tabularx}

\begin{tabularx}{\linewidth}{XV{3}X}
	\multicolumn{2}{l}{\textbf{Freizeit}} \\
	\bline
	Heute habe ich frei. & \\
\ro	Was \kjg{machen} \sbj am \kontext{Wochentag}? & Am \kontext{Wonchentag} \fillhere \\
	Haben Sie am \kontext{Wochentag} Zeit? & Nein, am \kontext{Wochentag} habe ich leider keine Zeit. \\
\ro	Haben Sie Zeit für \fillhere ? & Nein, Ich habe	leider keine Zeit für \fillhere . 
				Vielleicht können wir morgen zusammen \fillhere . \\
	Kannst du am \kontext{Wochentag}? & \\
	(Kannst du am \kontext{Wochentag} kommen?) & \\
	\hline
	Ich möchte bitte mit \kontext{Person} sprechen. & Ist es dringend? \\
\ro	Ja. Ist es heute möglich? & Also, am Vormittag \fillhere . Sie können heute leider nicht mit \sbj sprechen. \\
	Und \fillhere . Hat \sbj vielleicht \fillhere Zeit? & Nein. Sie können gerne \kontext{Zeit} wieder anrufen. \\
	Gut, ich rufe \kontext{Zeit} wieder an. & \\
	\multicolumn{2}{l}{}
\end{tabularx}
\end{discourse}

\begin{discourse}{Wohnen}{person}
\begin{tabularx}{\linewidth}{XV{3}X}
	\multicolumn{2}{l}{\textbf{Wohnung}} \\
	\bline
	\multicolumn{2}{l}{}
\end{tabularx}

\begin{tabularx}{\linewidth}{XV{3}X}
	\multicolumn{2}{l}{\textbf{Räume}} \\
	\bline
	\multicolumn{2}{l}{}
\end{tabularx}

\begin{tabularx}{\linewidth}{XV{3}X}
	\multicolumn{2}{l}{\textbf{Einrichtung/Möbel}} \\
	\bline
	\multicolumn{2}{l}{}
\end{tabularx}

\begin{tabularx}{\linewidth}{XV{3}X}
	\multicolumn{2}{l}{\textbf{Haushalt/technische Einrichtungen}} \\
	\bline
	\multicolumn{2}{l}{}
\end{tabularx}

\begin{tabularx}{\linewidth}{XV{3}X}
	\multicolumn{2}{l}{\textbf{Miete/Mietverhältnis}} \\
	\bline
	\multicolumn{2}{l}{}
\end{tabularx}

\begin{tabularx}{\linewidth}{XV{3}X}
	\multicolumn{2}{l}{\textbf{Wohnungswechsel}} \\
	\bline
	\multicolumn{2}{l}{}
\end{tabularx}
\end{discourse}

\begin{discourse}{Unsorted}{person}
\begin{tabularx}{\linewidth}{X}
	\textbf{Land} \\
	\bline
	\kontext{Land} ist \fillhere $km^2$ groß. \\
\ro	\kontext{Land} hat \fillhere Einwohner. \\
	\kontext{Land} hat \fillhere Bundesländer/Kantone. \\
\ro	\kontext{Land} hat \fillhere Millionenstädte/eine Millionenstadt. \\
	In der Hauptstadt \kontext{Stadt} wohnen \fillhere Menschen. \\
\ro	Die Vorwahl für \kontext{Land} ist \fillhere . \\
	\\
\end{tabularx}

\begin{tabularx}{\linewidth}{X}
	\textbf{Stadt} \\
	\bline
	\kontext{Stadt} ist über \kontext{Anzahl} Jahre alt. \\
	Die Stadt liegt in der Mitte von \kontext{Land}. \\
	In \kontext{Stadt} leben etwa \kontext{Anzalh} Menschen. \\
	In \kontext{Stadt} gibt es viele \fillhere . \\
	Die Stadt hat \fillhere . \\
	\\
\end{tabularx}

\begin{tabularx}{\linewidth}{XV{3}X}
	\multicolumn{2}{l}{\textbf{Balkendiagramm}} \\
	\bline
	Auf Platz \kontext{Position} liegt \fillhere . & \\
\ro	Insgesamt, \fillhere & \\
	\sbj \verbhere im Durchschnitt \fillhere . & \\
	\multicolumn{2}{l}{}
\end{tabularx}

\begin{tabularx}{\linewidth}{XV{3}X}
	\multicolumn{2}{l}{\textbf{Meinung}} \\
	\bline
	Wie \kjg{finden} \sbj \akk ? & \\
	\multicolumn{2}{l}{}
\end{tabularx}
\end{discourse}

\begin{discourse}{E-Mails}{e-mails}
\begin{tabularx}{\linewidth}{XV{3}X}
	\multicolumn{2}{l}{\textbf{E-Mails}} \\
	\bline
	Hallo \fillhere , & Sehr geehrte Damen und Herren , \\
	Mit freundlichen Grüßen, & \\
	\hline
	ich bin gerade in \kontext{Ort}. & \\
	\hline
	Bis bald & \\
	Lieber/Libebe \fillhere , & \\
	Schöne/Liebe Grüße, & \\
	\multicolumn{2}{l}{} \\
\end{tabularx}
\end{discourse}

%\begin{discourse}{Fragen Stadten}{}
\begin{enumerate}
	\item \begin{question}
			\item Wo ist \fillhere ?
		\end{question}
	\item \begin{answer}
		\item (Ich glaube,) \fillhere ist in \fillhere .
		\end{answer}
\end{enumerate}
\end{discourse}

%\begin{discourse}{Hobbys}{}
\begin{enumerate}
	\item \begin{question}
			\item Was machst du gern?
			\item Was machen Sie gern?
		\end{question}
	\item \begin{answer}
			\item Ich \verbhere gern \fillhere .
			\item Ich \verbhere nicht gern \fillhere .
			\item Ich \verbhere nicht so gern \fillhere .
		\end{answer}
	\item \begin{question}
			\item Ich auch.
			\item Interessant!
			\item Wirklich?
		\end{question}
	\item \begin{question}
			\item \verbhere \fillhere du/Sie (auch) gern?
		\end{question}
	\item \begin{answer}
			\item Ja, \fillhere \verbhere gern \fillhere .
			\item Nein, \fillhere \verbhere lieber \fillhere .
		\end{answer}
\end{enumerate}
\end{discourse}

%\begin{discourse}{Personen vorstellen}{}
\begin{enumerate}
	\item Das ist/sind \fillhere
	\item Er/sie/Sie kommt aus \fillhere
	\item Er/sie/Sie wohnt in \fillhere
	\item Er/sie/Sie spricht \fillhere
	\item Er/sie/Sie \verbhere gern \fillhere
	\item Er/sie/Sie ist \textit{Beruf}
	\item Er/sie/Sie arbeitet als \textit{Beruf}
\end{enumerate}
\end{discourse}

%\begin{discourse}{Berufe und Tätigkeiten}{}
\begin{enumerate}
	\item \begin{question}
			\item Was sind Sie von Beruf?
			\item Was machen Sie beruflich?
		\end{question}
	\item \begin{answer}
			\item Ich bin \textit{Beruf}.
			\item Ich arbeite als \textit{Beruf}.
		\end{answer}
	\item \begin{answer}
			\item Ich \verbhere \fillhere
			\item Ich bin beruflich oft in \fillhere .
			\item Ich bin beruflich hier.
		\end{answer}
\end{enumerate}
\end{discourse}

%\begin{discourse}{Was ist in \fillhere ?}{}
\begin{enumerate}
	\item \begin{question}
			\item Ist in \fillhere ein \fillhere ?
		\end{question}
	\item \begin{answer}
			\item Ja, in \fillhere ist ein \fillhere.
			\item Nein, in \fillhere ist kein \fillhere.
		\end{answer}
\end{enumerate}
\end{discourse}

%\begin{discourse}{Wie finden Sie \fillhere ?}{}
\begin{enumerate}
	\item \begin{question}
			\item Wie finden Sie \fillhere ?
		\end{question}
	\item \begin{answer}
			\item \sbj ist \adjhere .
		\end{answer}
		\begin{itemize}
			\item[allgemeine] schön
			\item[Buch] gut
			\item[Nahrungsmittel] lecker
		\end{itemize}
\end{enumerate}
\end{discourse}

%\begin{discourse}{Länder}{}
\begin{enumerate}
	\item \begin{answer}
		\item \textit{Land} ist \fillhere $km^2$ groß.
		\item \textit{Land} hat \fillhere Einwohner.
		\item In der Hauptstadt \textit{Land} wohnen \fillhere Menschen.
		\end{answer}
\end{enumerate}
\end{discourse}


%\begin{discourse}{Sprachen}{}
\begin{enumerate}
	\item \begin{answer}
			\item \fillhere Menschen sprechen \textit{Sprache} als Muttersprache.
		\end{answer}
\end{enumerate}
\end{discourse}

%\begin{discourse}{Chart}{}
\begin{enumerate}
	\item \begin{answer}
			\item \fillhere liegt auf Platz \textit{Position}
		\end{answer}
\end{enumerate}
\end{discourse}

%\begin{discourse}{Partnerinterview}{}
\begin{tabular}{|l|l|l|}
	\hline
	& Frage & Anwort \\
	\hline
	\multirow{2}{*}{Name} & \multirow{2}{*}{Wie heißen Sie?} & Ich bin \\
						  & & Ich heiße \\
	\hline
	Wohnort & Wo wohnen Sie? & Ich wohne in \fillhere \\
	\hline
	Adresse & Wie ist Ihre Adresse? & Meine Adresse ist \fillhere \\
	\hline
	Familienstand & Sind Sie verheiratet/ledig/geschieden? & Ja, Ich bin \fillhere \\
	\hline
	\multirow{3}{*}{Beruf} & Was ist Ihre Beruf? & Ich bin \fillhere \\
						   & Was sind Sie von Beruf? & Ich arbeite als \fillhere \\
						   & Was machen Sie beruflich? & \\
	\hline
	Tätigkeit & Was machen Sie als \fillhere ? & Ich \fillhere \\
	\hline
	Muttersprache & Was ist Ihre Muttersprache ? & Meine Muttersprache ist \fillhere \\
	\hline
	Sprache & Welche Sprachen sprechen Sie noch? & Ich sprechhe \fillhere \\
	\hline
	\multirow{2}{*}{Hobbys} & Was sind Ihre Hobbys? & \multirow{2}{*}{Ich \fillhere gern \fillhere .} \\
							& Was machen Sie gern? & \\
	\hline
\end{tabular}
\end{discourse}

%\begin{discourse}{Wie geht's?}{}
\begin{enumerate}
	\item \begin{question}
			\item Wie geht es dir?
			\item Wie geht es Ihnen?
		\end{question}
	\item \begin{answer}
			\item Danke, (auch) gut.
		\end{answer}
\end{enumerate}
\end{discourse}


%\begin{discourse}{Einkaufen/Gebrauchsartikel}{einkaufen}
\begin{tabularx}{\linewidth}{XV{3}X}
	\multicolumn{2}{l}{\textbf{Geschäfte}} \\
	\bline
	\multicolumn{2}{l}{}
\end{tabularx}

\begin{tabularx}{\linewidth}{XV{3}X}
	\multicolumn{2}{l}{\textbf{Preis/Bezahlen}} \\
	\bline
	\sbj bezahlen/zahlen für \fillhere \kontext{Preis}. & \\
	\hline
	\sbj \kontext{möchten} (gern) zahlen. & Das macht zusammen \kontext{Preis}. \\
	\hline
	Zusammen oder getrennt?		& Zusammen. \\
	\ro							& \fillhere und \fillhere, das macht \kontext{Preis}. \\
	\ro							& Das macht zusammaen \kontext{Preis}. \\
	Bitte. & Vielen Dank. Auf Wiedersehen. \\
	\hline
	& Bar. \\
	\multirow{-2}{\linewidth}{Wie zahlen Sie? Bar oder mit Kreditkarte?} & Mit Kreditkarte. \\
\ro	Was ist besonders teuer? & \\
\ro	Was kostet wenig Geld? & \\
	\multicolumn{2}{l}{}
\end{tabularx}

\begin{tabularx}{\linewidth}{XV{3}X}
	\multicolumn{2}{l}{\textbf{Lebensmittel}} \\
	\bline
	\multicolumn{2}{l}{}
\end{tabularx}

\begin{tabularx}{\linewidth}{XV{3}X}
	\multicolumn{2}{l}{\textbf{Kleidung}} \\
	\bline
	\multicolumn{2}{l}{}
\end{tabularx}
\end{discourse}


%\newpage

\bibliography{references} 
\bibliographystyle{ieeetr}

\end{document}
